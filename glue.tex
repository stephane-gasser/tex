\input webmac
% This program by D. E. Knuth is not copyrighted and can be used freely.
% It was written on 18 Dec 1981 and revised on 24 May 1991.

% Here is TeX material that gets inserted after \input webmac
\def\PASCAL{Pascal}
\font\eightrm=cmr8

\def\title{GLUE}
\def\topofcontents{\null
  \def\titlepage{F} % include headline on the contents page
  \def\rheader{\mainfont\hfil \contentspagenumber}
  \vfill
  \centerline{\titlefont Fixed-Point Glue Setting}
  \vfill}
\def\botofcontents{\vfill
  \centerline{\hsize 6in\baselineskip9pt
    \vbox{\eightrm\baselineskip9pt\noindent
    The preparation of this report
    was supported in part by the National Science
    Foundation under grants IST-7921977 and MCS-7723728;
    by Office of Naval Research grant N00014-81-K-0330;
    and by the IBM Corporation. `\TeX' is a
    trademark of the American Mathematical Society.}}}


\N1.  Introduction.
If \TeX\ is being implemented on a microcomputer that does 32-bit
addition and subtraction, but with multiplication and division restricted to
multipliers and divisors that are either powers of~2 or positive
integers less than~$2^{15}$, it can still do the computations associated
with the setting of glue in a suitable way. This program illustrates one
solution to the problem.

Another purpose of this program is to provide the first ``short'' example
of the use of \.{WEB}.

\fi

\M2. The program itself is written in standard \PASCAL. It begins with a
normal program header, most of which will be filled in with other parts of this
``web'' as we are ready to introduce them.

\Y\P\4\&{program}\1\  \37$\\{GLUE}(\\{input},\39\\{output})$;\6
\4\&{type} \37\X6:Types in the outer block\X\6
\4\&{var} \37\X8:Globals in the outer block\X\6
\4\&{procedure}\1\  \37\\{initialize};\C{this procedure gets things started}\6
\4\&{var} \37\X9:Local variables for initialization\X\2\6
\&{begin} \37\X10:Set initial values\X;\6
\&{end};\par
\fi

\M3. Here are two macros for common programming idioms.

\Y\P\D \37$\\{incr}(\#)\S\#\K\#+1$\C{increase a variable by unity}\par
\P\D \37$\\{decr}(\#)\S\#\K\#-1$\C{decrease a variable by unity}\par
\fi

\N4.  The problem and a solution.
We are concerned here with the ``setting of glue'' that occurs when a
\TeX\ box is being packaged. Let $x_1$, \dots,~$x_n$ be integers whose sum
$s=x_1+\cdots+x_n$ is positive, and let $t$ be another positive integer.
These $x_i$ represent scaled amounts of glue in units of sp (scaled
points), where one sp is $2^{-16}$ of a printer's point. The other
quantity $t$ represents the total by which the glue should stretch or
shrink. Following the conventions of \TeX82, we will assume that the
integers we deal with are less than $2^{31}$ in absolute value.

After the glue has been set, the actual amounts of incremental glue space
(in~sp) will be the integers $f(x_1)$, \dots,~$f(x_n)$, where $f$ is a
function that we wish to compute. We want $f(x)$ to be nearly proportional
to~$x$, and we also want the sum $f(x_1)+\cdots+f(x_n)$ to be nearly
equal to~$t$. If we were using floating-point arithmetic, we would simply
compute $f(x)\equiv(t/s)\cdot x$ and hope for the best; but the goal here
is to compute a suitable~$f$ using only the fixed-point arithmetic operations
of a typical ``16-bit microcomputer.''

The solution adopted here is to determine integers $a$, $b$, $c$ such that
$$f(x)=\bigl\lfloor 2^{-b}c\lfloor 2^{-a}x\rfloor\bigr\rfloor$$
if $x$ is nonnegative. Thus, we take $x$ and shift it right by $a$~bits,
then multiply by~$c$ (which is $2^{15}$ or less), and shift the product
right by $b$~bits. The quantities $a$, $b$, and~$c$ are to be chosen
so that this calculation doesn't cause overflow and so that $f(x_1)+\cdots
+f(x_n)$ is reasonably close to~$t$.

The following method is used to calculate $a$ and~$b$:
Suppose $$y=\max_{1\le i\le n}\vert x_i\vert\,.$$
Let $d$ and $e$ be the smallest integers such that $t<2^ds$ and $y<2^e$.
Since $s$ and~$t$ are less than~$2^{31}$, we have $-30\le d\le31$ and
$1\le e\le31$. An error message is given if $d+e\ge31$; in such a case
some $x_m$ has $\vert x_m\vert\ge 2^{e-1}$ and we are trying to change
$\vert x_m\vert$ to $\vert(t/s)x_m\vert\ge2^{d+e-2}\ge2^{30}$~sp, which
\TeX\ does not permit. (Consider, for example, the ``worst case'' situation
$x_1=2^{30}+1$, $x_2=-2^{30}$, $t=2^{31}-1$; surely we need not bother
trying to accommodate such anomalous combinations of values.) On the other
hand if $d+e\le31$, we set $a=e-16$ and $b=31-d-e$. Notice that this choice
of~$a$ guarantees that $\lfloor2^{-a}\vert x_i\vert\rfloor<2^{16}$. We will
choose~$c$ to be at most~$2^{15}$, so that the product will be less
than~$2^{31}$.

The computation of $c$ is the tricky part.
The ``ideal'' value for $c$ would be $\rho=2^{a+b}t/s$, since $f(x)$ should
be approximately $(t/s)\cdot x$. Furthermore it is better to have $c$ slightly
larger than~$\rho$, instead of slightly smaller, since the other operations
in $f(x)$ have a downward bias. Therefore we shall compute $c=\lceil\rho%
\rceil$.
Since $2^{a+b}t/s<2^{a+b+d}=2^{15}$, we have $c\le2^{15}$ as desired.

We want to compute $c=\lceil\rho\rceil$ exactly in all cases. There is no
difficulty if $s<2^{15}$, since $c$ can be computed directly using the
formula $c=\bigl\lfloor(2^{a+b}t+s-1)/s\bigr\rfloor$; overflow will not
occur since $2^{a+b}t<2^{15}s<2^{30}$.

Otherwise let $s=s_12^l+s_2$, where $2^{14}\le s_1<2^{15}$ and $0\le s_2<2^l$.
We will essentially carry out a long division. Let $t$ be ``normalized''
so that $2^{30}\le2^ht<2^{31}$ for some~$h$. Then we form the quotient and
remainder of $2^ht$ divided by~$s_1$,
$$ 2^ht=qs_1+r_0, \qquad 0\le r_0<s_1.$$
It follows that $2^{h+l}t-qs=2^lr_0-qs_2=r$, say. If $0\ge r>-s$ we have
$q=\lceil2^{h+l}t/s\rceil$; otherwise we can replace $(q,r)$ by
$(q\pm1,r\mp s)$ repeatedly until $r$ is in the correct range. It is not
difficult to prove that $q$ needs to be increased at most once and decreased
at most seven times, since $2^lr_0-qs_2<2^ls_1\le s$ and since
$qs_2/s\le(2^ht/s_1)(s_2/2^ls_1)<2^{31}/s_1^2\le8$. Finally, we have
$a+b-h-l=-1$ or~$-2$, since $2^{28+l}\le2^{14}s=2^{a+b+d-1}s\le2^{a+b}t<
2^{a+b+d}s=2^{15}s<2^{30+l}$ and $2^{30}\le2^ht<2^{31}$. Hence
$c=\lceil2^{a+b-h-l}q\rceil=\lceil{1\over2}q\rceil$ or~$\lceil{1\over4}q%
\rceil$.

An error analysis shows that these values of $a$, $b$, and $c$ work
satisfactorily, except in unusual cases where we wouldn't expect them to.
When $x\ge0$ we have
$$\eqalign{f(x)&=2^{-b}(2^{a+b}t/s+\theta_0)(2^{-a}x-\theta_1)-\theta_2\cr
&=(t/s)x+\theta_02^{-a-b}x-\theta_12^at/s-2^{-b}\theta_0\theta_1-\theta_2\cr}$$
where $0\le\theta_0,\theta_1,\theta_2<1$. Now $0\le\theta_02^{-a-b}x
<2^{e-a-b}=2^{d+e-15}$ and $0\le\theta_12^at/s<2^{a+d}=2^{d+e-16}$, and
the other two terms are negligible. Therefore $f(x_1)+\cdots+f(x_n)$ differs
from~$t$ by at most about $2^{d+e-15}n$. Since $2^{d+e}$ is larger than
$(t/s)y$, which is the largest stretching or shrinking of glue after expansion,
the error is at worst about $n/32000$ times as much as this, so it is quite
reasonable. For example, even if fill glue is being used to stretch
20 inches, the error will still be less than $1\over1600$ of an inch.

\fi

\M5. To sum up: Given the positive integers $s$, $t$, and $y$ as above, we
set $$a\gets\lfloor\lg y\rfloor-15,\qquad b\gets29-\lfloor\lg y\rfloor-
\lfloor\lg t/s\rfloor,\qquad\hbox{and}\qquad c\gets\lceil2^{a+b}t/s\rceil.$$
The implementation below shows how to do the job in \PASCAL\ without using
large numbers.

\fi

\M6. \TeX\ wants to have the glue-setting information in a 32-bit data type
called \\{glue\_ratio}. The \PASCAL\ implementation of \TeX82 has $\\{glue%
\_ratio}=\\{real}$, but alternative definitions of \\{glue\_ratio} are
explicitly allowed.

For our purposes we shall let \\{glue\_ratio} be a record that is packed with
three fields: The \\{a\_part} will hold the positive integer $\|a+16$, the
\\{b\_part} will hold the nonnegative integer~\|b, and the \\{c\_part} will
hold
the nonnegative integer~\|c. When the formulas above tell us to take
$\|b>30$, we might as well set $\|c\K0$ instead, because $\|f(\|x)$ will be
zero in all cases when $\|b>30$. Note that we have only about 25 bits of
information in all, so it should fit in 32 bits with ease.

\Y\P$\4\X6:Types in the outer block\X\S$\6
$\\{glue\_ratio}=$\1\5
\&{packed} \37\1\&{record} \37\\{a\_part}: \37$1\to31$;\C{the quantity $\|e=%
\|a+16$ in our derivation}\6
\4\\{b\_part}: \37$0\to30$;\C{the quantity \|b in our derivation}\6
\4\\{c\_part}: \37$0\to\O{100000}$;\C{the quantity \|c in our derivation}\2\6
\&{end};\2\6
$\\{scaled}=\\{integer}$;\C{this data type is used for quantities in sp units}%
\par
\U2.\fi

\M7. The real problem is to define the procedures that \TeX\ needs to
deal with such \\{glue\_ratio} values:
(a)~Given scaled numbers \|s, \|t, and~\|y as above, to compute the
corresponding \\{glue\_ratio}.
(b)~Given a nonnegative scaled number~\|x and a \\{glue\_ratio}~\|g, to
compute the scaled number~$\|f(\|x)$.
(c)~Given a \\{glue\_ratio}~\|g, to print out a decimal equivalent of
\|g for diagnostic purposes.

The procedures below can be incorporated into \TeX82 via a change file
without great difficulty. A few modifications will be needed, because
\TeX's \\{glue\_ratio} values can be negative in unusual cases---when the
amount of stretchability or shrinkability is less than zero. Negative
values in the \\{c\_part} will handle such problems, if proper care is
taken.  The error message below should either become a warning message
or a call to \TeX's \\{print\_err} routine; in the latter case, an
appropriate help message should be given, stating that glue cannot
stretch to more than 18~feet long, but that it's OK to proceed with
fingers crossed.

\fi

\N8. Glue multiplication.
The easiest procedure of the three just mentioned is the one that is
needed most often, namely, the computation of~$\|f(\|x)$.

\PASCAL\ doesn't have built-in binary shift commands or built-in
exponentiation,
although many computers do have this capability. Therefore our arithmetic
routines use an array called `\\{two\_to\_the}', containing powers of~two.
Divisions by powers of two are never done in the programs below when the
dividend is negative, so the operations can safely be replaced by right
shifts on machines for which this is most appropriate. (Contrary to popular
opinion, the operation `$\|x\mathbin{\&{div}}2$' is not the same as shifting %
\|x
right one binary place, on a machine with two's complement arithmetic,
when \|x is a negative odd integer. But division
{\it is\/} equivalent to shifting when \|x is nonnegative.)

\Y\P$\4\X8:Globals in the outer block\X\S$\6
\4\\{two\_to\_the}: \37\&{array} $[0\to30]$ \1\&{of}\5
\\{integer};\C{$\\{two\_to\_the}[k]=2^k$}\2\par
\As15\ET20.
\U2.\fi

\M9. \P$\X9:Local variables for initialization\X\S$\6
\4\|k: \37$1\to30$;\C{an index for initializing \\{two\_to\_the}}\par
\U2.\fi

\M10. \P$\X10:Set initial values\X\S$\6
$\\{two\_to\_the}[0]\K1$;\6
\&{for} $\|k\K1\mathrel{\&{to}}30$ \1\&{do}\5
$\\{two\_to\_the}[\|k]\K\\{two\_to\_the}[\|k-1]+\\{two\_to\_the}[\|k-1]$;\2\par
\U2.\fi

\M11. We will use the abbreviations \\{ga}, \\{gb}, and \\{gc} as convenient
alternatives to \PASCAL's \&{with} statement. The glue-multiplication
function \|f, which replaces several occurrences of the `\\{float}' macro
in \TeX82, is now easy to state:

\Y\P\D \37$\\{ga}\S\|g.\\{a\_part}$\par
\P\D \37$\\{gb}\S\|g.\\{b\_part}$\par
\P\D \37$\\{gc}\S\|g.\\{c\_part}$\par
\Y\P\4\&{function}\1\  \37$\\{glue\_mult}(\|x:\\{scaled};\,\35\|g:\\{glue%
\_ratio})$: \37\\{integer};\C{returns $\|f(\|x)$ as above, assuming that $\|x%
\G0$}\2\6
\&{begin} \37\&{if} $\\{ga}>16$ \1\&{then}\5
$\|x\K\|x\mathbin{\&{div}}\\{two\_to\_the}[\\{ga}-16]$\C{right shift by \|a
places}\6
\4\&{else} $\|x\K\|x\ast\\{two\_to\_the}[16-\\{ga}]$;\C{left shift by $-\|a$
places}\2\6
$\\{glue\_mult}\K(\|x\ast\\{gc})\mathbin{\&{div}}\\{two\_to\_the}[\\{gb}]$;%
\C{right shift by \|b places}\6
\&{end};\C{note that \|b may be as large as 30}\par
\fi

\N12. Glue setting.
The \\{glue\_fix} procedure computes \|a, \|b, and \|c by the method
explained above. \TeX\ does not normally compute the quantity~\|y, but
it could be made to do so without great difficulty.

This procedure replaces several occurrences of the `\\{unfloat}' macro in
\TeX82. It would be written as a function that returns a \\{glue\_ratio},
if \PASCAL\ would allow functions to produce records as values.

\Y\P\4\&{procedure}\1\  \37$\\{glue\_fix}(\|s,\39\|t,\39\|y:\\{scaled};\,\35%
\mathop{\&{var}}\|g:\\{glue\_ratio})$;\6
\4\&{var} \37$\|a,\39\|b,\39\|c$: \37\\{integer};\C{components of the desired
ratio}\6
$\|k,\39\|h$: \37\\{integer};\C{$30-\lfloor\lg s\rfloor$, $30-\lfloor\lg t%
\rfloor$}\6
\\{s0}: \37\\{integer};\C{original (unnormalized) value of \|s}\6
$\|q,\39\|r,\39\\{s1}$: \37\\{integer};\C{quotient, remainder, divisor}\6
\|w: \37\\{integer};\C{$2^l$, where $l=16-k$}\2\6
\&{begin} \37\X13:Normalize \|s, \|t, and \|y, computing \|a, \|k, and \|h\X;\6
\&{if} $\|t<\|s$ \1\&{then}\5
$\|b\K15-\|a-\|k+\|h$\ \&{else} $\|b\K14-\|a-\|k+\|h$;\2\6
\&{if} $(\|b<0)\V(\|b>30)$ \1\&{then}\6
\&{begin} \37\&{if} $\|b<0$ \1\&{then}\5
$\\{write\_ln}(\.{\'!\ Excessive\ glue.\'})$;\C{error message}\2\6
$\|b\K0$;\5
$\|c\K0$;\C{make $\|f(\|x)$ identically zero}\6
\&{end}\6
\4\&{else} \&{begin} \37\&{if} $\|k\G16$ \1\&{then}\C{easy case, $s_0<2^{15}$}\6
$\|c\K(\|t\mathbin{\&{div}}\\{two\_to\_the}[\|h-\|a-\|b]+\\{s0}-1)\mathbin{%
\&{div}}\\{s0}$\C{here $1\L\|h-\|a-\|b\L\|k-14\L16$}\6
\4\&{else} \X14:Compute \|c by long division\X;\2\6
\&{end};\2\6
$\\{ga}\K\|a+16$;\5
$\\{gb}\K\|b$;\5
$\\{gc}\K\|c$;\6
\&{end};\par
\fi

\M13. \P$\X13:Normalize \|s, \|t, and \|y, computing \|a, \|k, and \|h\X\S$\6
\&{begin} \37$\|a\K15$;\5
$\|k\K0$;\5
$\|h\K0$;\5
$\\{s0}\K\|s$;\6
\&{while} $\|y<\O{10000000000}$ \1\&{do}\C{\|y is known to be positive}\6
\&{begin} \37$\\{decr}(\|a)$;\5
$\|y\K\|y+\|y$;\6
\&{end};\2\6
\&{while} $\|s<\O{10000000000}$ \1\&{do}\C{\|s is known to be positive}\6
\&{begin} \37$\\{incr}(\|k)$;\5
$\|s\K\|s+\|s$;\6
\&{end};\2\6
\&{while} $\|t<\O{10000000000}$ \1\&{do}\C{\|t is known to be positive}\6
\&{begin} \37$\\{incr}(\|h)$;\5
$\|t\K\|t+\|t$;\6
\&{end};\2\6
\&{end}\C{now $2^{30}\le t=2^ht_0<2^{31}$ and $2^{30}\le s=2^ks_0<2^{31}$,
hence $d=k-h$ if $t/s<1$}\par
\U12.\fi

\M14. \P$\X14:Compute \|c by long division\X\S$\6
\&{begin} \37$\|w\K\\{two\_to\_the}[16-\|k]$;\5
$\\{s1}\K\\{s0}\mathbin{\&{div}}\|w$;\5
$\|q\K\|t\mathbin{\&{div}}\\{s1}$;\5
$\|r\K((\|t\mathbin{\&{mod}}\\{s1})\ast\|w)-((\\{s0}\mathbin{\&{mod}}\|w)\ast%
\|q)$;\6
\&{if} $\|r>0$ \1\&{then}\6
\&{begin} \37$\\{incr}(\|q)$;\5
$\|r\K\|r-\\{s0}$;\6
\&{end}\6
\4\&{else} \&{while} $\|r\L-\\{s0}$ \1\&{do}\6
\&{begin} \37$\\{decr}(\|q)$;\5
$\|r\K\|r+\\{s0}$;\6
\&{end};\2\2\6
\&{if} $\|a+\|b+\|k-\|h=15$ \1\&{then}\5
$\|c\K(\|q+1)\mathbin{\&{div}}2$\ \&{else} $\|c\K(\|q+3)\mathbin{\&{div}}4$;\2\6
\&{end}\par
\U12.\fi

\N15. Glue-set printing.
The last of the three procedures we need is \\{print\_gr}, which displays a
\\{glue\_ratio} in symbolic decimal form. Before constructing such a procedure,
we shall consider some simpler routines, copying them from an early
draft of the program \TeX82.

\Y\P\D \37$\\{unity}\S\O{200000}$\C{$2^{16}$, represents 1.0000}\par
\Y\P$\4\X8:Globals in the outer block\X\mathrel{+}\S$\6
\4\\{dig}: \37\&{array} $[0\to15]$ \1\&{of}\5
$0\to9$;\C{for storing digits}\2\par
\fi

\M16. An array of digits is printed out by \\{print\_digs}.

\Y\P\4\&{procedure}\1\  \37$\\{print\_digs}(\|k:\\{integer})$;\C{prints $%
\\{dig}[\|k-1]$ \dots $\\{dig}[0]$}\2\6
\&{begin} \37\&{while} $\|k>0$ \1\&{do}\6
\&{begin} \37$\\{decr}(\|k)$;\5
$\\{write}(\\{chr}(\\{ord}(\.{\'0\'})+\\{dig}[\|k]))$;\6
\&{end};\2\6
\&{end};\par
\fi

\M17. A nonnegative integer is printed out by \\{print\_int}.

\Y\P\4\&{procedure}\1\  \37$\\{print\_int}(\|n:\\{integer})$;\C{prints an
integer in decimal form}\6
\4\&{var} \37\|k: \37$0\to12$;\C{index to current digit; we assume that $0\le
n<10^{12}$}\2\6
\&{begin} \37$\|k\K0$;\6
\1\&{repeat} \37$\\{dig}[\|k]\K\|n\mathbin{\&{mod}}10$;\5
$\|n\K\|n\mathbin{\&{div}}10$;\5
$\\{incr}(\|k)$;\6
\4\&{until}\5
$\|n=0$;\2\6
$\\{print\_digs}(\|k)$;\6
\&{end};\par
\fi

\M18. And here is a procedure to print a nonnegative \\{scaled} number.

\Y\P\4\&{procedure}\1\  \37$\\{print\_scaled}(\|s:\\{scaled})$;\C{prints a
scaled real, truncated to four digits}\6
\4\&{var} \37\|k: \37$0\to3$;\C{index to current digit of the fraction part}\2\6
\&{begin} \37$\\{print\_int}(\|s\mathbin{\&{div}}\\{unity})$;\C{print the
integer part}\6
$\|s\K((\|s\mathbin{\&{mod}}\\{unity})\ast10000)\mathbin{\&{div}}\\{unity}$;\6
\&{for} $\|k\K0\mathrel{\&{to}}3$ \1\&{do}\6
\&{begin} \37$\\{dig}[\|k]\K\|s\mathbin{\&{mod}}10$;\5
$\|s\K\|s\mathbin{\&{div}}10$;\6
\&{end};\2\6
$\\{write}(\.{\'.\'})$;\5
$\\{print\_digs}(4)$;\6
\&{end};\par
\fi

\M19. Now we're ready to print a \\{glue\_ratio}. Since the effective
multiplier
is $2^{-a-b}c$, we will display the scaled integer $2^{16-a-b}c$, taking
care to print something special if this quantity is terribly large.

\Y\P\4\&{procedure}\1\  \37$\\{print\_gr}(\|g:\\{glue\_ratio})$;\C{prints a
glue multiplier}\6
\4\&{var} \37\|j: \37$-29\to31$;\C{the amount to shift \|c}\2\6
\&{begin} \37$\|j\K32-\\{ga}-\\{gb}$;\6
\&{while} $\|j>15$ \1\&{do}\6
\&{begin} \37$\\{write}(\.{\'2x\'})$;\5
$\\{decr}(\|j)$;\C{indicate multiples of 2 for BIG cases}\6
\&{end};\2\6
\&{if} $\|j<0$ \1\&{then}\5
$\\{print\_scaled}(\\{gc}\mathbin{\&{div}}\\{two\_to\_the}[-\|j])$\C{shift
right}\6
\4\&{else} $\\{print\_scaled}(\\{gc}\ast\\{two\_to\_the}[\|j])$;\C{shift left}%
\2\6
\&{end};\par
\fi

\N20.  The driver program.
In order to test these routines, we will assume that the \\{input} file
contains a sequence of test cases, where each test case consists of the
integer numbers $t$, $x_1$, \dots,~$x_n$, 0. The final test case should
be followed by an additional zero.

\Y\P$\4\X8:Globals in the outer block\X\mathrel{+}\S$\6
\4\|x: \37\&{array} $[1\to1000]$ \1\&{of}\5
\\{scaled};\C{the $x_i$}\2\6
\4\|t: \37\\{scaled};\C{the desired total}\6
\4\|m: \37\\{integer};\C{the test case number}\par
\fi

\M21. Each case will be processed by the following routine, which assumes
that \|t has already been read.

\Y\P\4\&{procedure}\1\  \37\\{test};\C{processes the next data set, given \|t
and~\|m}\6
\4\&{var} \37\|n: \37$0\to1000$;\C{the number of items}\6
\|k: \37$0\to1000$;\C{runs through the items}\6
\|y: \37\\{scaled};\C{$\max_{1\le i\le n}\vert x_i\vert$}\6
\|g: \37\\{glue\_ratio};\C{the computed glue multiplier}\6
\|s: \37\\{scaled};\C{the sum $x_1+\cdots+x_n$}\6
\\{ts}: \37\\{scaled};\C{the sum $f(x_1)+\cdots+f(x_n)$}\2\6
\&{begin} \37$\\{write\_ln}(\.{\'Test\ data\ set\ number\ \'},\39\|m:1,\39\.{%
\':\'})$;\5
\X22:Read $x_1,\ldots,x_n$\X;\6
\X23:Compute \|s and \|y\X;\6
\&{if} $\|s\L0$ \1\&{then}\5
$\\{write\_ln}(\.{\'Invalid\ data\ (nonpositive\ sum);\ this\ set\ rejected.%
\'})$\6
\4\&{else} \&{begin} \37\X24:Compute \|g and print it\X;\6
\X25:Print the values of $x_i$, $f(x_i)$, and the totals\X;\6
\&{end};\2\6
\&{end};\par
\fi

\M22. \P$\X22:Read $x_1,\ldots,x_n$\X\S$\6
\&{begin} \37$\|n\K0$;\6
\1\&{repeat} \37$\\{incr}(\|n)$;\5
$\\{read}(\|x[\|n])$;\6
\4\&{until}\5
$\|x[\|n]=0$;\2\6
$\\{decr}(\|n)$;\6
\&{end}\par
\U21.\fi

\M23. \P$\X23:Compute \|s and \|y\X\S$\6
\&{begin} \37$\|s\K0$;\5
$\|y\K0$;\6
\&{for} $\|k\K1\mathrel{\&{to}}\|n$ \1\&{do}\6
\&{begin} \37$\|s\K\|s+\|x[\|k]$;\6
\&{if} $\|y<\\{abs}(\|x[\|k])$ \1\&{then}\5
$\|y\K\\{abs}(\|x[\|k])$;\2\6
\&{end};\2\6
\&{end}\par
\U21.\fi

\M24. \P$\X24:Compute \|g and print it\X\S$\6
\&{begin} \37$\\{glue\_fix}(\|s,\39\|t,\39\|y,\39\|g)$;\C{set \|g, perhaps
print an error message}\6
$\\{write}(\.{\'\ \ Glue\ ratio\ is\ \'})$;\5
$\\{print\_gr}(\|g)$;\5
$\\{write\_ln}(\.{\'\ (\'},\39\\{ga}-16:1,\39\.{\',\'},\39\\{gb}:1,\39\.{\',%
\'},\39\\{gc}:1,\39\.{\')\'})$;\6
\&{end}\par
\U21.\fi

\M25. \P$\X25:Print the values of $x_i$, $f(x_i)$, and the totals\X\S$\6
\&{begin} \37$\\{ts}\K0$;\6
\&{for} $\|k\K1\mathrel{\&{to}}\|n$ \1\&{do}\6
\&{begin} \37$\\{write}(\|x[\|k]:20)$;\6
\&{if} $\|x[\|k]\G0$ \1\&{then}\5
$\|y\K\\{glue\_mult}(\|x[\|k],\39\|g)$\6
\4\&{else} $\|y\K-\\{glue\_mult}(-\|x[\|k],\39\|g)$;\2\6
$\\{write\_ln}(\|y:15)$;\5
$\\{ts}\K\\{ts}+\|y$;\6
\&{end};\2\6
$\\{write\_ln}(\.{\'\ Totals\'},\39\|s:13,\39\\{ts}:15,\39\.{\'\ (versus\ \'},%
\39\|t:1,\39\.{\')\'})$;\6
\&{end}\par
\U21.\fi

\M26. Here is the main program.

\Y\P\&{begin} \37\\{initialize};\5
$\|m\K1$;\5
$\\{read}(\|t)$;\6
\&{while} $\|t>0$ \1\&{do}\6
\&{begin} \37\\{test};\5
$\\{incr}(\|m)$;\5
$\\{read}(\|t)$;\6
\&{end};\2\6
\&{end}.\par
\fi

\N27. Index. Here are the section numbers where various identifiers are used in
the
program, and where various topics are discussed.





\fi


\inx
\:\|{a}, \[12].
\:\\{a\_part}, \[6], 11.
\:\\{abs}, 23.
\:\|{b}, \[12].
\:\\{b\_part}, \[6], 11.
\:\|{c}, \[12].
\:\\{c\_part}, \[6], 7, 11.
\:\\{chr}, 16.
\:\\{decr}, \[3], 13, 14, 16, 19, 22.
\:\\{dig}, \[15], 16, 17, 18.
\:{error analysis}, 4.
\:{error message}, 7, 12.
\:\\{float}, 11.
\:\|{g}, \[11], \[12], \[19], \[21].
\:\\{ga}, \[11], 12, 19, 24.
\:\\{gb}, \[11], 12, 19, 24.
\:\\{gc}, \[11], 12, 19, 24.
\:\\{GLUE}, \[2].
\:\\{glue\_fix}, \[12], 24.
\:\\{glue\_mult}, \[11], 25.
\:\\{glue\_ratio}, \[6], 7, 11, 12, 15, 19, 21.
\:\|{h}, \[12].
\:{hairy mathematics}, 4.
\:\\{incr}, \[3], 13, 14, 17, 22, 26.
\:\\{initialize}, \[2], 26.
\:\\{input}, \[2], 20.
\:\\{integer}, 6, 8, 11, 12, 16, 17, 20.
\:\|{j}, \[19].
\:\|{k}, \[9], \[12], \[16], \[17], \[18].
\:\|{m}, \[20].
\:{main program}, 26.
\:\|{n}, \[17], \[21].
\:\\{ord}, 16.
\:\\{output}, \[2].
\:\\{print\_digs}, \[16], 17, 18.
\:\\{print\_err}, 7.
\:\\{print\_gr}, 15, \[19], 24.
\:\\{print\_int}, \[17], 18.
\:\\{print\_scaled}, \[18], 19.
\:{program header}, 2.
\:\|{q}, \[12].
\:\|{r}, \[12].
\:\\{read}, 22, 26.
\:\\{real}, 6.
\:\|{s}, \[12], \[21].
\:\\{scaled}, \[6], 11, 12, 18, 20, 21.
\:\\{s0}, \[12], 13, 14.
\:\\{s1}, \[12], 14.
\:\|{t}, \[12], \[20].
\:\\{test}, \[21], 26.
\:\\{ts}, \[21], 25.
\:\\{two\_to\_the}, \[8], 9, 10, 11, 12, 14, 19.
\:\\{unfloat}, 12.
\:\\{unity}, \[15], 18.
\:\|{w}, \[12].
\:\\{write}, 16, 18, 19, 24, 25.
\:\\{write\_ln}, 12, 21, 24, 25.
\:\|{x}, \[11], \[20].
\:\|{y}, \[12].
\fin
\:\X14:Compute \|c by long division\X
\U12.
\:\X24:Compute \|g and print it\X
\U21.
\:\X23:Compute \|s and \|y\X
\U21.
\:\X8, 15, 20:Globals in the outer block\X
\U2.
\:\X9:Local variables for initialization\X
\U2.
\:\X13:Normalize \|s, \|t, and \|y, computing \|a, \|k, and \|h\X
\U12.
\:\X25:Print the values of $x_i$, $f(x_i)$, and the totals\X
\U21.
\:\X22:Read $x_1,\ldots,x_n$\X
\U21.
\:\X10:Set initial values\X
\U2.
\:\X6:Types in the outer block\X
\U2.
\con
